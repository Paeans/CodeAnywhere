\documentclass[a4paper]{article}
%\usepackage[utf8]{inputenc}
\usepackage[a4paper]{geometry}
\usepackage{amsfonts}

\title{Homework of Cloud Computing}
\author{\Large \textbf{2015216027 \hspace{2cm} Pan Gaofeng}}
\date{December 01 2015}

\begin{document}

\Large
\maketitle

\section{ Q: Please describe the service models of Cloud Computing}

\hspace{1cm}\textbf{A:} There are 3 service models of Cloud Computing.

\begin{itemize}
\item \textbf{IaaS}: Infrastructure as a Service

In this service model, the cloud service providers provide to the customers are fundamental computing resources like storage, network, computer etc. The consumers can deploy and run arbitrary software, which include operating systems and applications. 

With this type of service model, consumers can specify what they want the system to do. Many cloud providers have this type of service model, like Amazon Web Service, Windows Azure and Aliyun.

\vspace{2mm} 

\item \textbf{PaaS}: Platform as a Service

PaaS service model provide to the consumers the platforms on which they can deploy consumer-created or acquired applications. Always the providers will also provide development tools like programming language, libraries and services.

With this type of service model, consumers don't need to take care of the application run-time environment, and they can put focus on the design and development of application. And this service model will help consumers to save a lot of time which will be wasted on building development and run-time environment.

\vspace{2mm} 

\item \textbf{SaaS}: Software as a Service

SaaS service model provide to the consumers the ability to use some specific applications which are build and deployed on the cloud infrastructure by the provider. Consumers choose the applications they want to use depending on their purpose.

As the applications are running on the provider's cloud infrastructure, consumers can (almost) use the applications at anytime and anywhere. And consumers do not need to run complex software on their own computer, then will reduce the requirements of local system. Services like Gmail, Google Docs and Evernote is this type of model.

\vspace{2mm} 

\end{itemize}

\vspace{5mm} 

\section{ Q: Please describe the deployment models of cloud computing}

\hspace{1cm}\textbf{A:} There are 3 deployment models of Cloud Computing.
\begin{itemize}
\item \textbf{Public}

Public cloud open to anyone who pays for the services. Some services like Amazon S3, Amazon EC2 and Google App Engine are this type of deployment model.

\vspace{2mm} 

\item \textbf{Private}

Private cloud open to a specific group of people. Usually this group of people are belong to a company or a community. Using this model is considering the safety and usability.

\vspace{2mm} 

\item \textbf{Hybrid}

Hybrid cloud model is a composition of two or more distinct cloud infrastructures. Those infrastructures are bound together by standardized or proprietary technology. With this model, data and application portability can be proved.

\vspace{2mm} 

\end{itemize}

\end{document}

